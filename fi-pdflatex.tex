%%%%%%%%%%%%%%%%%%%%%%%%%%%%%%%%%%%%%%%%%%%%%%%%%%%%%%%%%%%%%%%%%%%%
%% I, the copyright holder of this work, release this work into the
%% public domain. This applies worldwide. In some countries this may
%% not be legally possible; if so: I grant anyone the right to use
%% this work for any purpose, without any conditions, unless such
%% conditions are required by law.
%%%%%%%%%%%%%%%%%%%%%%%%%%%%%%%%%%%%%%%%%%%%%%%%%%%%%%%%%%%%%%%%%%%%

\documentclass[
  digital,     %% The `digital` option enables the default options for the
               %% digital version of a document. Replace with `printed`
               %% to enable the default options for the printed version
               %% of your thesis.
%%  color,       %% Uncomment these lines (by removing the %% at the
%%               %% beginning) to use color in the printed version of your
%%               %% document
  oneside,     %% The `oneside` option enables one-sided typesetting,
               %% which is preferred if you are only going to submit a
               %% digital version of your thesis. Replace with `twoside`
               %% for double-sided typesetting if you are planning to
               %% also print your thesis. For double-sided typesetting,
               %% use at least 120 g/m² paper to prevent show-through.
  nosansbold,  %% The `nosansbold` option prevents the use of the
               %% sans-serif type face for bold text. Replace with `sansbold` to use sans-serif type face for bold text.
  nocolorbold, %% The `nocolorbold` option disables the usage of the
               %% blue color for bold text, instead using black. Replace with `colorbold` to use blue for bold text.
  lof,         %% The `lof` option prints the List of Figures. Replace
               %% with `nolof` to hide the List of Figures.
  lot,         %% The `lot` option prints the List of Tables. Replace
               %% with `nolot` to hide the List of Tables.
]{fithesis4}
%% The following section sets up the locales used in the thesis.
\usepackage[resetfonts]{cmap} %% We need to load the T2A font encoding
\usepackage[T1,T2A]{fontenc}  %% to use the Cyrillic fonts with Russian texts.
\usepackage[
  main=english, %% By using `czech` or `slovak` as the main locale
                %% instead of `english`, you can typeset the thesis
                %% in either Czech or Slovak, respectively.
  english, german, czech, slovak %% The additional keys allow
]{babel}        %% foreign texts to be typeset as follows:
%%
%%   \begin{otherlanguage}{german}  ... \end{otherlanguage}
%%   \begin{otherlanguage}{czech}   ... \end{otherlanguage}
%%   \begin{otherlanguage}{slovak}  ... \end{otherlanguage}
%%
%%
%% The following section sets up the metadata of the thesis.
\thesissetup{
    date        = \the\year/\the\month/\the\day,
    university  = mu,
    faculty     = fi,
    type        = bc,
    department  = Department of Machine Learning and Data Processing,
    author      = {Bc. Maxmilián Šeffer},
    gender      = m,
    advisor     = {doc. Ing. Václav Oujezský, Ph.D.},
    title       = {Smart Timesheet App},
    TeXtitle    = {Smart Timesheet Application},
    keywords    = {TO DO keywords},
    TeXkeywords = {keyword1, keyword2, \ldots},
    abstract    = {%
        This thesis explores the potential of audio transcription technology to simplify time tracking for employees working in the field. The goal is to develop a cloud-based application that allows employees to log their working hours by recording audio descriptions of their tasks, specifying the client and project. This recorded speech is then transcribed into text and converted into structured database entries. The research evaluates the reliability and practicality of this approach in real-world conditions.
        
        In addition to automated transcription, the application provides features for replaying audio recordings, manually editing and managing entries, and exporting data for managerial review. By simplifying the logging process, the solution aims to minimize forgetfulness of employees and improve accuracy of work record while offering employees an efficient way to track their work.
    },
    thanks      = {%
      I would like to thank doc. Ing. Václav Oujezský, Ph.D., for his guidance and leadership throughout this thesis. My thanks also go to Bc. Ondřej Zelinka, with whom I collaborated as he was working simultaneously on a web version of the front-end. I am also grateful to Roman Kalous for his support and cooperation on behalf of the company.
    },
    bib         = example.bib,
    %% Remove the following line to use the JVS 2018 faculty logo.
    facultyLogo = fithesis-fi,
}
\usepackage{makeidx}      %% The `makeidx` package contains
\makeindex                %% helper commands for index typesetting.
%% These additional packages are used within the document:
\usepackage{paralist} %% Compact list environments
\usepackage{amsmath}  %% Mathematics
\usepackage{amsthm}
\usepackage{amsfonts}
\usepackage{url}      %% Hyperlinks
\usepackage{markdown} %% Lightweight markup
\usepackage{listings} %% Source code highlighting
\lstset{
  basicstyle      = \ttfamily,
  identifierstyle = \color{black},
  keywordstyle    = \color{blue},
  keywordstyle    = {[2]\color{cyan}},
  keywordstyle    = {[3]\color{olive}},
  stringstyle     = \color{teal},
  commentstyle    = \itshape\color{magenta},
  breaklines      = true,
}
\usepackage{floatrow} %% Putting captions above tables
\floatsetup[table]{capposition=top}
\usepackage[babel]{csquotes} %% Context-sensitive quotation marks

\begin{document}
%% The \chapter* command can be used to produce unnumbered chapters:
\chapter*{Introduction}
%% Unlike \chapter, \chapter* does not update the headings and does not
%% enter the chapter to the table of contents. If we want correct
%% headings and a table of contents entry, we must add them manually:
\markright{\textsc{Introduction}}
\addcontentsline{toc}{chapter}{Introduction}
TODO

\begin{otherlanguage}{czech}
TODO
\end{otherlanguage}

\chapter{Requirements Analysis}

\section{Introduction}

In order to develop a practical and effective cloud-based audio transcription system for time tracking, a comprehensive requirements analysis is essential. This phase establishes a clear understanding of the functionalities and constraints that the software must adhere to, ensuring alignment with user needs and business objectives. The analysis focuses on defining user roles, identifying functional and non-functional requirements, and outlining key technical considerations.

The goal of this analysis is to create a structured framework that guides the development of the Smart Timesheet Application. Through discussions with potential users, stakeholders, and domain experts, the analysis highlights both explicit and implicit needs, as well as potential challenges that may arise during implementation.

\section{Users}

The primary users of the Smart Timesheet Application are mainly field employees. These users frequently switch between tasks and clients throughout the day. Their primary need is a quick, reliable, and intuitive method to log their working hours and task details in real-time without requiring a traditional computer interface. They benefit from hands-free audio logging that ensures accuracy and minimizes forgotten details.

\section{Functional Requirements}

The functional requirements define the core capabilities of the Smart Timesheet Application, ensuring that it meets user expectations and performs its intended tasks efficiently. These requirements specify the system's behavior in terms of services, tasks, and functions that users can perform.  

To clearly outline the expected functionality, the requirements are presented in the form of use cases. Each use case describes a specific interaction between the user and the system, defining the prerequisites, the sequence of steps, and the expected outcome.  

The use cases are not listed by priority but instead grouped at the same level of abstraction for clarity. While some use cases could be decomposed into more detailed subcases, this document keeps them at a high level to maintain readability. Detailed business logic, rules, and dependencies will be enforced through integration tests in the system’s codebase, avoiding redundant specifications in this document.  

The core functional requirements of the Smart Timesheet Application revolve around efficient, voice-enabled time tracking, cloud-based data storage, and secure authentication. The application must accommodate hands-free logging while ensuring that users can review, edit, and manage their work records with minimal friction. Additionally, robust data security mechanisms must be in place to guarantee data integrity and user privacy.

\subsection{Adding Work Records via Audio}

\textbf{Actors:} Employee  

\textbf{Prerequisites:} User is authenticated  

\textbf{Description:}  
Employees can record audio descriptions of their work, specifying the client, project, and task details. The system transcribes the audio and converts it into a structured work entry.  

\textbf{Steps:}  
\begin{enumerate}
    \item Employee initiates an audio recording in the application.
    \item Employee verbally describes the work performed.
    \item System uploads the recording to the cloud.
    \item Transcription service processes the audio and converts it into text.
    \item System structures the transcribed data into a work record.
    \item Work entry is stored in the cloud-based database.
    \item Employee receives a notification upon transcription completion.
\end{enumerate}

\subsection{Adding Work Records Manually}

\textbf{Actors:} Employee  

\textbf{Prerequisites:} User is authenticated  

\textbf{Description:}  
Employees can manually enter work records if they prefer not to use audio transcription.  

\textbf{Steps:}  
\begin{enumerate}
    \item Employee navigates to the manual work entry interface.
    \item Employee inputs details such as client, project, and task description.
    \item Work entry is saved to the cloud database.
\end{enumerate}

\subsection{Reviewing and Editing Work Entries}

\textbf{Actors:} Employee  

\textbf{Prerequisites:} User is authenticated, work entries exist  

\textbf{Description:}  
Employees can review and edit their past work records, including both manually entered and transcribed entries.  

\textbf{Steps:}  
\begin{enumerate}
    \item Employee accesses the work log history.
    \item Employee selects a work entry to review.
    \item Employee can replay the original audio recording (if available).
    \item If necessary, employee edits the transcription or manually entered details.
    \item Updated entry is saved to the cloud database.
\end{enumerate}

\subsection{Searching and Filtering Work Entries}

\textbf{Actors:} Employee  

\textbf{Prerequisites:} User is authenticated, work entries exist  

\textbf{Description:}  
Employees can search for and filter work logs based on specific criteria such as date, project, or client.  

\textbf{Steps:}  
\begin{enumerate}
    \item Employee accesses the search and filter interface.
    \item Employee selects criteria for filtering (e.g., date range, client, project).
    \item System retrieves and displays the matching work entries.
\end{enumerate}

\subsection{Cloud-Based Data Storage and Organization}

\textbf{Actors:} System  

\textbf{Prerequisites:} Work entries exist  

\textbf{Description:}  
Work records are securely stored in a cloud-based database. Employees can access their records from any authenticated device.  

\textbf{Steps:}  
\begin{enumerate}
    \item Work entries are saved in a structured format in the cloud.
    \item System ensures data is organized for efficient retrieval.
    \item Employees can easily view, sort, and access their stored records.
\end{enumerate}

\subsection{User Authentication and Access Control}

\textbf{Actors:} Employee  

\textbf{Prerequisites:} None  

\textbf{Description:}  
Authentication and access control ensure that only authorized users can log in and access their respective data.  

\textbf{Steps:}  
\begin{enumerate}
    \item Employee attempts to log in via Microsoft authentication.
    \item System verifies credentials through Firebase Authentication.
    \item System checks the employee’s email domain.
    \item If the domain is authorized, the employee is granted access to their records.
    \item System ensures that employees from different domains cannot view or modify each other's records.
\end{enumerate}

\subsection{Offline Audio Logging}

\textbf{Actors:} Employee  

\textbf{Prerequisites:} User is authenticated, no internet connection available  

\textbf{Description:}  
Employees can record work descriptions offline, and the system will sync and process the recordings once an internet connection is available.  

\textbf{Steps:}  
\begin{enumerate}
    \item Employee records an audio entry while offline.
    \item System stores the recording locally on the device.
    \item Once an internet connection is restored, the system automatically uploads and processes the recording.
    \item Transcribed entry is added to the work log.
\end{enumerate}

\subsection{Receiving Notifications}

\textbf{Actors:} Employee  

\textbf{Prerequisites:} User is authenticated, relevant event occurs  

\textbf{Description:}  
Employees receive notifications when important events occur, such as transcription completion or record updates.  

\textbf{Steps:}  
\begin{enumerate}
    \item System detects an event requiring a notification (e.g., transcription complete).
    \item System generates a notification message.
    \item Employee receives a push notification or email alert.
\end{enumerate}

\section{Non-Functional Requirements}

While functional requirements define what the system must do, non-functional requirements ensure that the system operates effectively under real-world conditions. These requirements specify qualities such as performance, security, usability, and scalability, ensuring that the application remains reliable, efficient, and user-friendly over time.  

The Smart Timesheet Application is designed to accommodate employees working in various environments, including fieldwork where hands-free operation and offline functionality may be necessary. Additionally, as a cloud-based solution, the system must provide seamless data synchronization and high availability, while ensuring that sensitive work records remain secure and private.  

The following sections outline the essential characteristics that the system must meet to provide a smooth, secure, and scalable experience.

\subsection{Performance and Reliability}  
The transcription process does not need to be instant, but users must be notified upon completion so they can review and verify the results. The system must handle multiple concurrent users and organizations without performance degradation.  

While a constant internet connection is not required, users must be able to record audio offline. The recorded audio should be queued and automatically processed when the device reconnects to the internet. Additionally, the system should ensure that no data is lost in case of connection interruptions.

\subsection{Usability and Accessibility}  
The user interface should be designed with simplicity and ease of use in mind, requiring minimal training. Hands-free operation should be a priority, with voice commands and simple touch interactions allowing users to log work efficiently without needing extensive manual input.  

The system should also provide clear visual and auditory feedback to confirm actions. Text should be minimized for core functionality but available for advanced configurations and settings.

\subsection{Security and Compliance}  
All user data must be securely stored and transmitted using encryption. The application must comply with GDPR and other relevant data protection regulations, ensuring that users’ personal and work-related information is not exposed or misused.  

Additionally, access control mechanisms must prevent unauthorized users from accessing or modifying work records.

\subsection{Scalability}  
The system must be designed to support a growing number of users and organizations. As adoption increases, performance should remain stable, and the backend infrastructure must be capable of handling increased storage and processing demands without requiring significant downtime or maintenance.

\subsection{Cross-Platform Support}  
The application must be available on both Android and iOS platforms, ensuring that employees can access and use the system regardless of their device. The mobile application should maintain consistent functionality and usability across different devices and operating system versions.

\subsection{Logging and Audit Trails}  
The system must maintain detailed audit logs for key operations, including the creation and modification of work records. Specifically, changes to work entries, authentication events, and offline-to-online data synchronization must be logged.  

These logs should provide administrators and developers with insights into potential issues, user errors, or security concerns.

\subsection{Failure Handling and Reporting}  
The application must include mechanisms for detecting and reporting failures. If an entry fails to synchronize due to network issues or a transcription error, the user must be notified, and the failure must be logged for further debugging.  

Complete system failures should be automatically reported to the development team, enabling proactive maintenance and issue resolution.
    
\shorthandoff{-}
\begin{markdown}[
  citationNbsps,
  citations,
  definitionLists,
  fencedCode,
  hashEnumerators,
  inlineNotes,
  notes,
  pipeTables,
  rawAttribute,
  tableCaptions,
]
   
\end{markdown}
\shorthandon{-}

\chapter{Market Analysis}

\section{Introduction}
The need for efficient time-tracking solutions has led to the development of various tools aimed at helping businesses and employees manage work hours effectively. Many existing solutions rely on manual entry, mobile applications, or automated tracking features. However, there are still gaps in usability, accuracy, and hands-free operation, particularly for field employees who require a more seamless and automated way to log their time.

\section{Existing Solutions}

Several time-tracking applications currently serve businesses, each with strengths and limitations:

\begin{itemize}
\item \textbf{Clockify}: A widely used time-tracking tool that provides manual entry, timers, and integration with various project management tools. While powerful, it requires manual input or interaction with a timer, which may not be ideal for field employees who need a hands-free option.
\item \textbf{Toggl Track}: Offers intuitive time tracking with automation features, but still relies on user interaction to start and stop time logs. It lacks built-in voice transcription for work entry.
\item \textbf{Hubstaff}: Includes GPS tracking and automated timesheets, making it useful for remote teams, but does not focus on voice-based logging.
\item \textbf{TSheets by QuickBooks}: Provides detailed time-tracking features and payroll integration but requires manual input or GPS tracking, with no emphasis on voice-based automation.
\end{itemize}

\section{Strengths and weaknesses of existing solutions}

\begin{itemize}  
    \item \textbf{Strengths:} Most existing solutions provide cloud-based tracking, integrations with payroll and project management tools, and mobile accessibility. In addition, these applications are often highly optimized and feature rich, ensuring a streamlined user experience. However, many of their most valuable features are restricted behind paywalls, necessitating costly subscriptions. Moreover, businesses frequently lack control over the application's functionality and long-term availability, making them dependent on third-party providers. Organizations require a time-tracking solution that offers reliability and stability without the risk of unexpected feature limitations, forced upgrades, or service discontinuation.  

    Another significant limitation is the absence of dedicated audio-to-work-entry functionality in existing solutions. Although some applications permit audio attachments as supplementary notes, they do not provide automated transcription or structured conversion of voice recordings into work entries. As a result, users must rely on manual data entry or timers, which may be inefficient for field employees who require a seamless and hands-free method of logging work hours.  

    \item \textbf{Weaknesses:} The majority of existing solutions rely on manual input, timers, or GPS tracking, which may be impractical for employees working in dynamic field environments. Although some applications incorporate voice-based logging, these implementations are generally limited to note taking rather than full transcription and structured work entry generation. Consequently, such solutions function more as auxiliary documentation tools rather than complete work-logging systems, failing to address the specific needs of professionals who require accurate and automated time tracking.

\end{itemize}

\shorthandoff{-}
\begin{markdown}[
  citationNbsps,
  citations,
  definitionLists,
  fencedCode,
  hashEnumerators,
  inlineNotes,
  notes,
  pipeTables,
  rawAttribute,
  tableCaptions,
]
\end{markdown}
\shorthandon{-}

\chapter{Design and Prototyping}
\shorthandoff{-}
\begin{markdown}[
  citationNbsps,
  citations,
  definitionLists,
  fencedCode,
  hashEnumerators,
  inlineNotes,
  notes,
  pipeTables,
  rawAttribute,
  tableCaptions,
]
  % TODO: Present early design iterations, wireframes, user flows, and feedback from stakeholders.
\end{markdown}
\shorthandon{-}

\chapter{Implementation Overview}
\shorthandoff{-}
\begin{markdown}[
  citationNbsps,
  citations,
  definitionLists,
  fencedCode,
  hashEnumerators,
  inlineNotes,
  notes,
  pipeTables,
  rawAttribute,
  tableCaptions,
]
  % TODO: Overview of key technologies and architectural decisions in the development process.
\end{markdown}
\shorthandon{-}

\chapter{Development Process and Challenges}
\shorthandoff{-}
\begin{markdown}[
  citationNbsps,
  citations,
  definitionLists,
  fencedCode,
  hashEnumerators,
  inlineNotes,
  notes,
  pipeTables,
  rawAttribute,
  tableCaptions,
]
  % TODO: Describe the development methodology (e.g., Agile), challenges faced during development, and how they were addressed.
\end{markdown}
\shorthandon{-}

\chapter{Testing and Validation}
\shorthandoff{-}
\begin{markdown}[
  citationNbsps,
  citations,
  definitionLists,
  fencedCode,
  hashEnumerators,
  inlineNotes,
  notes,
  pipeTables,
  rawAttribute,
  tableCaptions,
]
  % TODO: Document testing phases, including unit tests, integration tests, and user feedback.
\end{markdown}
\shorthandon{-}

\chapter{Conclusion and Future Work}
\shorthandoff{-}
\begin{markdown}[
  citationNbsps,
  citations,
  definitionLists,
  fencedCode,
  hashEnumerators,
  inlineNotes,
  notes,
  pipeTables,
  rawAttribute,
  tableCaptions,
]
  % TODO: Summarize key takeaways and suggest potential directions for future development or improvement.
\end{markdown}
\shorthandon{-}

\chapter{Example Chapter}
\shorthandoff{-}
\begin{markdown}[
  citationNbsps,
  citations,
  definitionLists,
  fencedCode,
  hashEnumerators,
  inlineNotes,
  notes,
  pipeTables,
  rawAttribute,
  tableCaptions,
]

Here is an example of a citation to avoid warnings: ~[@borgman03, p. 123].

\end{markdown}
\shorthandon{-}

\appendix %% Start the appendices.
\chapter{Appendix A: Additional Information}
TODO appendices

\end{document}
