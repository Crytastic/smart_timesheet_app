%%%%%%%%%%%%%%%%%%%%%%%%%%%%%%%%%%%%%%%%%%%%%%%%%%%%%%%%%%%%%%%%%%%%
%% I, the copyright holder of this work, release this work into the
%% public domain. This applies worldwide. In some countries this may
%% not be legally possible; if so: I grant anyone the right to use
%% this work for any purpose, without any conditions, unless such
%% conditions are required by law.
%%%%%%%%%%%%%%%%%%%%%%%%%%%%%%%%%%%%%%%%%%%%%%%%%%%%%%%%%%%%%%%%%%%%

\documentclass[
  digital,     %% The `digital` option enables the default options for the
               %% digital version of a document. Replace with `printed`
               %% to enable the default options for the printed version
               %% of your thesis.
%%  color,       %% Uncomment these lines (by removing the %% at the
%%               %% beginning) to use color in the printed version of your
%%               %% document
  oneside,     %% The `oneside` option enables one-sided typesetting,
               %% which is preferred if you are only going to submit a
               %% digital version of your thesis. Replace with `twoside`
               %% for double-sided typesetting if you are planning to
               %% also print your thesis. For double-sided typesetting,
               %% use at least 120 g/m² paper to prevent show-through.
  nosansbold,  %% The `nosansbold` option prevents the use of the
               %% sans-serif type face for bold text. Replace with `sansbold` to use sans-serif type face for bold text.
  nocolorbold, %% The `nocolorbold` option disables the usage of the
               %% blue color for bold text, instead using black. Replace with `colorbold` to use blue for bold text.
  lof,         %% The `lof` option prints the List of Figures. Replace
               %% with `nolof` to hide the List of Figures.
  lot,         %% The `lot` option prints the List of Tables. Replace
               %% with `nolot` to hide the List of Tables.
]{fithesis4}
%% The following section sets up the locales used in the thesis.
\usepackage[resetfonts]{cmap} %% We need to load the T2A font encoding
\usepackage[T1,T2A]{fontenc}  %% to use the Cyrillic fonts with Russian texts.
\usepackage[
  main=english, %% By using `czech` or `slovak` as the main locale
                %% instead of `english`, you can typeset the thesis
                %% in either Czech or Slovak, respectively.
  english, german, czech, slovak %% The additional keys allow
]{babel}        %% foreign texts to be typeset as follows:
%%
%%   \begin{otherlanguage}{german}  ... \end{otherlanguage}
%%   \begin{otherlanguage}{czech}   ... \end{otherlanguage}
%%   \begin{otherlanguage}{slovak}  ... \end{otherlanguage}
%%
%%
%% The following section sets up the metadata of the thesis.
\thesissetup{
    date        = \the\year/\the\month/\the\day,
    university  = mu,
    faculty     = fi,
    type        = bc,
    department  = Department of Machine Learning and Data Processing,
    author      = {Bc. Maxmilián Šeffer},
    gender      = m,
    advisor     = {doc. Ing. Václav Oujezský, Ph.D.},
    title       = {Smart Timesheet App},
    TeXtitle    = {Smart Timesheet Application},
    keywords    = {TO DO keywords},
    TeXkeywords = {keyword1, keyword2, \ldots},
    abstract    = {%
        This thesis explores the potential of audio transcription technology to simplify time tracking for employees working in the field. The goal is to develop a cloud-based application that allows employees to log their working hours by recording audio descriptions of their tasks, specifying the client and project. This recorded speech is then transcribed into text and converted into structured database entries. The research evaluates the reliability and practicality of this approach in real-world conditions.
        
        In addition to automated transcription, the application provides features for replaying audio recordings, manually editing and managing entries, and exporting data for managerial review. By simplifying the logging process, the solution aims to minimize forgetfulness of employees and improve accuracy of work record while offering employees an efficient way to track their work.
    },
    thanks      = {%
      I would like to thank doc. Ing. Václav Oujezský, Ph.D., for his guidance and leadership throughout this thesis. My thanks also go to Bc. Ondřej Zelinka, with whom I collaborated as he was working simultaneously on a web version of the front-end. I am also grateful to Roman Kalous for his support and cooperation on behalf of the company.
    },
    bib         = example.bib,
    %% Remove the following line to use the JVS 2018 faculty logo.
    facultyLogo = fithesis-fi,
}
\usepackage{makeidx}      %% The `makeidx` package contains
\makeindex                %% helper commands for index typesetting.
%% These additional packages are used within the document:
\usepackage{paralist} %% Compact list environments
\usepackage{amsmath}  %% Mathematics
\usepackage{amsthm}
\usepackage{amsfonts}
\usepackage{url}      %% Hyperlinks
\usepackage{markdown} %% Lightweight markup
\usepackage{listings} %% Source code highlighting
\lstset{
  basicstyle      = \ttfamily,
  identifierstyle = \color{black},
  keywordstyle    = \color{blue},
  keywordstyle    = {[2]\color{cyan}},
  keywordstyle    = {[3]\color{olive}},
  stringstyle     = \color{teal},
  commentstyle    = \itshape\color{magenta},
  breaklines      = true,
}
\usepackage{floatrow} %% Putting captions above tables
\floatsetup[table]{capposition=top}
\usepackage[babel]{csquotes} %% Context-sensitive quotation marks

\begin{document}
%% The \chapter* command can be used to produce unnumbered chapters:
\chapter*{Introduction}
%% Unlike \chapter, \chapter* does not update the headings and does not
%% enter the chapter to the table of contents. If we want correct
%% headings and a table of contents entry, we must add them manually:
\markright{\textsc{Introduction}}
\addcontentsline{toc}{chapter}{Introduction}
TODO

\begin{otherlanguage}{czech}
TODO
\end{otherlanguage}

\chapter{Requirements Analysis}

\section{Introduction}

In order to develop a practical and effective cloud-based audio transcription system for time tracking, a comprehensive requirements analysis is essential. This phase establishes a clear understanding of the functionalities and constraints that the software must adhere to, ensuring alignment with user needs and business objectives. The analysis focuses on defining user roles, identifying functional and non-functional requirements, and outlining key technical considerations.

The goal of this analysis is to create a structured framework that guides the development of the Smart Timesheet Application. Through discussions with potential users, stakeholders, and domain experts, the analysis highlights both explicit and implicit needs, as well as potential challenges that may arise during implementation.

\section{Users}

The primary users of the Smart Timesheet Application are mainly field employees. These users frequently switch between tasks and clients throughout the day. Their primary need is a quick, reliable, and intuitive method to log their working hours and task details in real-time without requiring a traditional computer interface. They benefit from hands-free audio logging that ensures accuracy and minimizes forgotten details.

\section{Functional Requirements}

The functional requirements define the core capabilities of the system, ensuring that it meets user expectations and performs its intended tasks efficiently.

\begin{itemize} \item \textbf{Adding work records:} Users can add work records via audio recording or manually. For audio, they can record descriptions of tasks, specifying the client, project, description, and other details. The app converts the audio into a structured work record and saves it to a cloud-based database. Manual entry is also supported.

\item \textbf{Review of entries:} Users can review audio recordings and manually edit work records.

\item \textbf{Data Storage and Organization:} Entries are securely stored in a cloud-based database. The entries are simple to view and search.

\item \textbf{User Authentication and Access Control:} Authentication is handled via Firebase Authentication, with login restricted to Microsoft accounts. Access is granted based on the user's email domain, ensuring that unauthorized domains are restricted. Data isolation ensures that users from different domains cannot view or alter each other's records.
\end{itemize}

\section{Non-Functional Requirements}

Beyond core functionalities, non-functional requirements ensure that the system is reliable, secure, and user-friendly.

\begin{itemize} \item \textbf{Performance and Reliability:} The transcription process does not need to be instant, but users should be notified when it is finished. Upon notification, users can review and verify the transcription. The system must handle multiple concurrent users and organizations without performance degradation. While the app does not need to be fully offline, users should be able to record audio without an active internet connection. The entry should automatically be processed and synchronized once a connection is available.

\item \textbf{Usability and Accessibility:} The user interface should be intuitive and require minimal training. Voice commands and simple touch interactions should be prioritized for hands-free use. Text should be minimized for common usage, though more detailed configurations or preferences may require additional text.

\item \textbf{Security and Compliance:} All data stored and transmitted must be encrypted to protect user privacy. The system should comply with GDPR and other relevant data protection regulations.

\item \textbf{Scalability:} The system should support an increasing number of users and organizations.
\end{itemize}

    
\shorthandoff{-}
\begin{markdown}[
  citationNbsps,
  citations,
  definitionLists,
  fencedCode,
  hashEnumerators,
  inlineNotes,
  notes,
  pipeTables,
  rawAttribute,
  tableCaptions,
]
   
\end{markdown}
\shorthandon{-}

\chapter{Market Analysis}


\section{Introduction}
The need for efficient time-tracking solutions has led to the development of various tools aimed at helping businesses and employees manage work hours effectively. Many existing solutions rely on manual entry, mobile applications, or automated tracking features. However, there are still gaps in usability, accuracy, and hands-free operation, particularly for field employees who require a more seamless and automated way to log their time.

\section{Existing Solutions}

Several time-tracking applications currently serve businesses, each with strengths and limitations:

\begin{itemize}
\item \textbf{Clockify}: A widely used time-tracking tool that provides manual entry, timers, and integration with various project management tools. While powerful, it requires manual input or interaction with a timer, which may not be ideal for field employees who need a hands-free option.
\item \textbf{Toggl Track}: Offers intuitive time tracking with automation features, but still relies on user interaction to start and stop time logs. It lacks built-in voice transcription for work entry.
\item \textbf{Hubstaff}: Includes GPS tracking and automated timesheets, making it useful for remote teams, but does not focus on voice-based logging.
\item \textbf{TSheets by QuickBooks}: Provides detailed time-tracking features and payroll integration but requires manual input or GPS tracking, with no emphasis on voice-based automation.
\end{itemize}

\section{Strengths and weaknesses of existing solutions}

\begin{itemize}  
    \item \textbf{Strengths:} Most existing solutions provide cloud-based tracking, integrations with payroll and project management tools, and mobile accessibility. In addition, these applications are often highly optimized and feature rich, ensuring a streamlined user experience. However, many of their most valuable features are restricted behind paywalls, necessitating costly subscriptions. Moreover, businesses frequently lack control over the application's functionality and long-term availability, making them dependent on third-party providers. Organizations require a time-tracking solution that offers reliability and stability without the risk of unexpected feature limitations, forced upgrades, or service discontinuation.  

    Another significant limitation is the absence of dedicated audio-to-work-entry functionality in existing solutions. Although some applications permit audio attachments as supplementary notes, they do not provide automated transcription or structured conversion of voice recordings into work entries. As a result, users must rely on manual data entry or timers, which may be inefficient for field employees who require a seamless and hands-free method of logging work hours.  

    \item \textbf{Weaknesses:} The majority of existing solutions rely on manual input, timers, or GPS tracking, which may be impractical for employees working in dynamic field environments. Although some applications incorporate voice-based logging, these implementations are generally limited to note taking rather than full transcription and structured work entry generation. Consequently, such solutions function more as auxiliary documentation tools rather than complete work-logging systems, failing to address the specific needs of professionals who require accurate and automated time tracking.

\end{itemize}

\shorthandoff{-}
\begin{markdown}[
  citationNbsps,
  citations,
  definitionLists,
  fencedCode,
  hashEnumerators,
  inlineNotes,
  notes,
  pipeTables,
  rawAttribute,
  tableCaptions,
]
\end{markdown}
\shorthandon{-}

\chapter{Design and Prototyping}
\shorthandoff{-}
\begin{markdown}[
  citationNbsps,
  citations,
  definitionLists,
  fencedCode,
  hashEnumerators,
  inlineNotes,
  notes,
  pipeTables,
  rawAttribute,
  tableCaptions,
]
  % TODO: Present early design iterations, wireframes, user flows, and feedback from stakeholders.
\end{markdown}
\shorthandon{-}

\chapter{Implementation Overview}
\shorthandoff{-}
\begin{markdown}[
  citationNbsps,
  citations,
  definitionLists,
  fencedCode,
  hashEnumerators,
  inlineNotes,
  notes,
  pipeTables,
  rawAttribute,
  tableCaptions,
]
  % TODO: Overview of key technologies and architectural decisions in the development process.
\end{markdown}
\shorthandon{-}

\chapter{Development Process and Challenges}
\shorthandoff{-}
\begin{markdown}[
  citationNbsps,
  citations,
  definitionLists,
  fencedCode,
  hashEnumerators,
  inlineNotes,
  notes,
  pipeTables,
  rawAttribute,
  tableCaptions,
]
  % TODO: Describe the development methodology (e.g., Agile), challenges faced during development, and how they were addressed.
\end{markdown}
\shorthandon{-}

\chapter{Testing and Validation}
\shorthandoff{-}
\begin{markdown}[
  citationNbsps,
  citations,
  definitionLists,
  fencedCode,
  hashEnumerators,
  inlineNotes,
  notes,
  pipeTables,
  rawAttribute,
  tableCaptions,
]
  % TODO: Document testing phases, including unit tests, integration tests, and user feedback.
\end{markdown}
\shorthandon{-}

\chapter{Conclusion and Future Work}
\shorthandoff{-}
\begin{markdown}[
  citationNbsps,
  citations,
  definitionLists,
  fencedCode,
  hashEnumerators,
  inlineNotes,
  notes,
  pipeTables,
  rawAttribute,
  tableCaptions,
]
  % TODO: Summarize key takeaways and suggest potential directions for future development or improvement.
\end{markdown}
\shorthandon{-}

\chapter{Example Chapter}
\shorthandoff{-}
\begin{markdown}[
  citationNbsps,
  citations,
  definitionLists,
  fencedCode,
  hashEnumerators,
  inlineNotes,
  notes,
  pipeTables,
  rawAttribute,
  tableCaptions,
]

Here is an example of a citation to avoid warnings: ~[@borgman03, p. 123].

\end{markdown}
\shorthandon{-}

\appendix %% Start the appendices.
\chapter{Appendix A: Additional Information}
TODO appendices

\end{document}
