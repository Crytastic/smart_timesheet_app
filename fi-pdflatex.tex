%%%%%%%%%%%%%%%%%%%%%%%%%%%%%%%%%%%%%%%%%%%%%%%%%%%%%%%%%%%%%%%%%%%%
%% I, the copyright holder of this work, release this work into the
%% public domain. This applies worldwide. In some countries this may
%% not be legally possible; if so: I grant anyone the right to use
%% this work for any purpose, without any conditions, unless such
%% conditions are required by law.
%%%%%%%%%%%%%%%%%%%%%%%%%%%%%%%%%%%%%%%%%%%%%%%%%%%%%%%%%%%%%%%%%%%%

\documentclass[
  digital,     %% The `digital` option enables the default options for the
               %% digital version of a document. Replace with `printed`
               %% to enable the default options for the printed version
               %% of your thesis.
%%  color,       %% Uncomment these lines (by removing the %% at the
%%               %% beginning) to use color in the printed version of your
%%               %% document
  oneside,     %% The `oneside` option enables one-sided typesetting,
               %% which is preferred if you are only going to submit a
               %% digital version of your thesis. Replace with `twoside`
               %% for double-sided typesetting if you are planning to
               %% also print your thesis. For double-sided typesetting,
               %% use at least 120 g/m² paper to prevent show-through.
  nosansbold,  %% The `nosansbold` option prevents the use of the
               %% sans-serif type face for bold text. Replace with `sansbold` to use sans-serif type face for bold text.
  nocolorbold, %% The `nocolorbold` option disables the usage of the
               %% blue color for bold text, instead using black. Replace with `colorbold` to use blue for bold text.
  lof,         %% The `lof` option prints the List of Figures. Replace
               %% with `nolof` to hide the List of Figures.
  lot,         %% The `lot` option prints the List of Tables. Replace
               %% with `nolot` to hide the List of Tables.
]{fithesis4}
%% The following section sets up the locales used in the thesis.
\usepackage[resetfonts]{cmap} %% We need to load the T2A font encoding
\usepackage[T1,T2A]{fontenc}  %% to use the Cyrillic fonts with Russian texts.
\usepackage[
  main=english, %% By using `czech` or `slovak` as the main locale
                %% instead of `english`, you can typeset the thesis
                %% in either Czech or Slovak, respectively.
  english, german, czech, slovak %% The additional keys allow
]{babel}        %% foreign texts to be typeset as follows:
%%
%%   \begin{otherlanguage}{german}  ... \end{otherlanguage}
%%   \begin{otherlanguage}{czech}   ... \end{otherlanguage}
%%   \begin{otherlanguage}{slovak}  ... \end{otherlanguage}
%%
%%
%% The following section sets up the metadata of the thesis.
\thesissetup{
    date        = \the\year/\the\month/\the\day,
    university  = mu,
    faculty     = fi,
    type        = bc,
    department  = Department of Machine Learning and Data Processing,
    author      = {Bc. Maxmilián Šeffer},
    gender      = m,
    advisor     = {doc. Ing. Václav Oujezský, Ph.D.},
    title       = {Smart Timesheet App},
    TeXtitle    = {Smart Timesheet Application},
    keywords    = {TO DO keywords},
    TeXkeywords = {keyword1, keyword2, \ldots},
    abstract    = {%
        This thesis explores the potential of audio transcription technology to simplify time tracking for employees working in the field. The goal is to develop a cloud-based application that allows employees to log their working hours by recording audio descriptions of their tasks, specifying the client and project. This recorded speech is then transcribed into text and converted into structured database entries. The research evaluates the reliability and practicality of this approach in real-world conditions.
        
        In addition to automated transcription, the application provides features for replaying audio recordings, manually editing and managing entries, and exporting data for managerial review. By simplifying the logging process, the solution aims to minimize forgetfulness of employees and improve accuracy of work record while offering employees an efficient way to track their work.
    },
    thanks      = {%
      I would like to thank doc. Ing. Václav Oujezský, Ph.D., for his guidance and leadership throughout this thesis. My thanks also go to Bc. Ondřej Zelinka, with whom I collaborated as he was working simultaneously on a web version of the front-end. I am also grateful to Roman Kalous for his support and cooperation on behalf of the company.
    },
    bib         = example.bib,
    %% Remove the following line to use the JVS 2018 faculty logo.
    facultyLogo = fithesis-fi,
}
\usepackage{makeidx}      %% The `makeidx` package contains
\makeindex                %% helper commands for index typesetting.
%% These additional packages are used within the document:
\usepackage{paralist} %% Compact list environments
\usepackage{amsmath}  %% Mathematics
\usepackage{amsthm}
\usepackage{amsfonts}
\usepackage{url}      %% Hyperlinks
\usepackage{markdown} %% Lightweight markup
\usepackage{listings} %% Source code highlighting
\lstset{
  basicstyle      = \ttfamily,
  identifierstyle = \color{black},
  keywordstyle    = \color{blue},
  keywordstyle    = {[2]\color{cyan}},
  keywordstyle    = {[3]\color{olive}},
  stringstyle     = \color{teal},
  commentstyle    = \itshape\color{magenta},
  breaklines      = true,
}
\usepackage{floatrow} %% Putting captions above tables
\floatsetup[table]{capposition=top}
\usepackage[babel]{csquotes} %% Context-sensitive quotation marks

\begin{document}
%% The \chapter* command can be used to produce unnumbered chapters:
\chapter*{Introduction}
%% Unlike \chapter, \chapter* does not update the headings and does not
%% enter the chapter to the table of contents. If we want correct
%% headings and a table of contents entry, we must add them manually:
\markright{\textsc{Introduction}}
\addcontentsline{toc}{chapter}{Introduction}
TODO

\begin{otherlanguage}{czech}
TODO
\end{otherlanguage}

\chapter{Market Analysis}
\shorthandoff{-}
\begin{markdown}[
  citationNbsps,
  citations,
  definitionLists,
  fencedCode,
  hashEnumerators,
  inlineNotes,
  notes,
  pipeTables,
  rawAttribute,
  tableCaptions,
]
  % TODO: A detailed analysis of current time-tracking solutions, strengths, and weaknesses. 
\end{markdown}
\shorthandon{-}

\chapter{Project Goals and Scope}
\shorthandoff{-}
\begin{markdown}[
  citationNbsps,
  citations,
  definitionLists,
  fencedCode,
  hashEnumerators,
  inlineNotes,
  notes,
  pipeTables,
  rawAttribute,
  tableCaptions,
]
  % TODO: Define the specific goals of the Smart Timesheet App. What sets it apart from existing solutions?
\end{markdown}
\shorthandon{-}

\chapter{Functional and Non-Functional Requirements}
\shorthandoff{-}
\begin{markdown}[
  citationNbsps,
  citations,
  definitionLists,
  fencedCode,
  hashEnumerators,
  inlineNotes,
  notes,
  pipeTables,
  rawAttribute,
  tableCaptions,
]
  % TODO: Detail the functional (e.g., features, user stories) and non-functional requirements (e.g., performance, security) for the app.
\end{markdown}
\shorthandon{-}

\chapter{Design and Prototyping}
\shorthandoff{-}
\begin{markdown}[
  citationNbsps,
  citations,
  definitionLists,
  fencedCode,
  hashEnumerators,
  inlineNotes,
  notes,
  pipeTables,
  rawAttribute,
  tableCaptions,
]
  % TODO: Present early design iterations, wireframes, user flows, and feedback from stakeholders.
\end{markdown}
\shorthandon{-}

\chapter{Implementation Overview}
\shorthandoff{-}
\begin{markdown}[
  citationNbsps,
  citations,
  definitionLists,
  fencedCode,
  hashEnumerators,
  inlineNotes,
  notes,
  pipeTables,
  rawAttribute,
  tableCaptions,
]
  % TODO: Overview of key technologies and architectural decisions in the development process.
\end{markdown}
\shorthandon{-}

\chapter{Development Process and Challenges}
\shorthandoff{-}
\begin{markdown}[
  citationNbsps,
  citations,
  definitionLists,
  fencedCode,
  hashEnumerators,
  inlineNotes,
  notes,
  pipeTables,
  rawAttribute,
  tableCaptions,
]
  % TODO: Describe the development methodology (e.g., Agile), challenges faced during development, and how they were addressed.
\end{markdown}
\shorthandon{-}

\chapter{Testing and Validation}
\shorthandoff{-}
\begin{markdown}[
  citationNbsps,
  citations,
  definitionLists,
  fencedCode,
  hashEnumerators,
  inlineNotes,
  notes,
  pipeTables,
  rawAttribute,
  tableCaptions,
]
  % TODO: Document testing phases, including unit tests, integration tests, and user feedback.
\end{markdown}
\shorthandon{-}

\chapter{Conclusion and Future Work}
\shorthandoff{-}
\begin{markdown}[
  citationNbsps,
  citations,
  definitionLists,
  fencedCode,
  hashEnumerators,
  inlineNotes,
  notes,
  pipeTables,
  rawAttribute,
  tableCaptions,
]
  % TODO: Summarize key takeaways and suggest potential directions for future development or improvement.
\end{markdown}
\shorthandon{-}

\chapter{Example Chapter}
\shorthandoff{-}
\begin{markdown}[
  citationNbsps,
  citations,
  definitionLists,
  fencedCode,
  hashEnumerators,
  inlineNotes,
  notes,
  pipeTables,
  rawAttribute,
  tableCaptions,
]

Here is an example of a citation to avoid warnings: ~[@borgman03, p. 123].

\end{markdown}
\shorthandon{-}

\appendix %% Start the appendices.
\chapter{Appendix A: Additional Information}
TODO appendices

\end{document}
